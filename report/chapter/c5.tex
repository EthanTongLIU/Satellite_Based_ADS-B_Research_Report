% !TEX root = ../document.tex

\chapter{总结}

通过以上调研,ADS-B 技术拥有成本低、精度误差小、监视能力强的特点,适用于高密度飞行区域的空中交通服务,有效解决和避免了传统监视方式的种种弊端,优越性不言而喻,必将成为未来空管监视的主要手段。

目前,根据中国在 ICAO 提交的相关报告,ADS-B 技术在中国还未全面展开,但是正在稳步推进。成长期的中国航空运输业面临空域范围的限制,机队规模扩大,要求空管设施进一步改造和完善,是加速 ADS-B  技术发展的重要时期。

目前国际上,丹麦、德国、美国、加拿大和中国等国家(某些合资企业是跨国的)相继展开了星基 ADS-B 项目的实验和系统部署,但是目前只有 Aireon 真正依托第二代铱星卫星部署完成了星基 ADS-B 系统并已经开始试运营和提供服务,Aireon 已经走在了世界的前列。不过。该项技术目前还存在诸多问题亟待解决,等未来全球电磁环境进一步恶化后,该项技术的发展前景可能又是另外一番景象。

我国应开展相应的卫星网络建设,加强基于卫星的 ADS-B 系统和基于陆基的 ADS-B 系统同步发展,同时兼顾其他监视手段,构建“空-天-地”一体的全球监视系统。此外,ADS-B 技术的推广应用,可能涉及航空公司的软件更新、改装机载设备、调整地面设施结构、标准制定和运行认证等,因而需要各方有力配合,整体推进,才能早日实现 ADS-B 的全面应用,为实现建设民航强国奠定基础\upcite{z1}。